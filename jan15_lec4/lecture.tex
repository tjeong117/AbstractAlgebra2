\documentclass{article}
\usepackage{amsmath, amsthm, amssymb, amsfonts}
\usepackage{thmtools}
\usepackage{graphicx}
\usepackage{setspace}
\usepackage{cancel}
\usepackage{geometry}
\usepackage{float}
\usepackage{hyperref}
\usepackage[utf8]{inputenc}
\usepackage[english]{babel}
\usepackage{framed}
\usepackage[dvipsnames]{xcolor}
\usepackage{tcolorbox}
\usepackage{amsmath}
\usepackage{array}
\usepackage{tikz} 
\usepackage{multirow}
\usepackage{tcolorbox}
\usepackage{xcolor}
\usepackage{tikz-cd}
\usepackage{xcolor}
\usepackage{wasysym}

% Define a new color for the example box



\colorlet{LightGray}{White!90!Periwinkle}
\colorlet{LightOrange}{Orange!15}
\colorlet{LightGreen}{Green!15}

\newcommand{\HRule}[1]{\rule{\linewidth}{#1}}

\colorlet{LightGray}{black!10}
\colorlet{LightOrange}{orange!15}
\colorlet{LightGreen}{green!15}
\colorlet{LightBlue}{blue!15}
\colorlet{LightCyan}{cyan!15}



\declaretheoremstyle[name=Theorem,]{thmsty}
\declaretheorem[style=thmsty,numberwithin=section]{theorem}
\usepackage{tcolorbox} % Add missing package
\tcolorboxenvironment{theorem}{colback=LightGray}

\declaretheoremstyle[name=Definition,]{thmsty}
\declaretheorem[style=thmsty,numberwithin=section]{definition}
\tcolorboxenvironment{definition}{colback=LightBlue}

\declaretheoremstyle[name=Proposition,]{prosty}
\declaretheorem[style=prosty,numberlike=theorem]{proposition}
\tcolorboxenvironment{proposition}{colback=LightOrange}


\declaretheoremstyle[name=Example,]{prosty}
\declaretheorem[style=prosty,numberlike=theorem]{example}
\tcolorboxenvironment{example}{colback=LightOrange}

\declaretheoremstyle[name=Axiom,]{prcpsty}
\declaretheorem[style=prcpsty,numberlike=theorem]{axiom}
\tcolorboxenvironment{axiom}{colback=LightGreen}

\declaretheoremstyle[name=Lemma,]{prcpsty}
\declaretheorem[style=prcpsty,numberlike=theorem]{lemma}
\tcolorboxenvironment{lemma}{colback=LightCyan}





\setstretch{1.2}
\geometry{
    textheight=9in,
    textwidth=5.5in,
    top=1in,
    headheight=12pt,
    headsep=25pt,
    footskip=30pt
}

% ------------------------------------------------------------------------------

\begin{document}

% ------------------------------------------------------------------------------
% Cover Page and ToC
% ------------------------------------------------------------------------------

\title{ \normalsize \textsc{}
		\\ [2.0cm]
		\HRule{1.5pt} \\
		\LARGE \textbf{\uppercase{Lecture 4}}
		\HRule{2.0pt} \\ [0.6cm] \LARGE{}
		}

\date{\today}
\author{\textbf{Author} \\ 
		Tom Jeong
        }

\maketitle

\tableofcontents
\newpage

% ------------------------------------------------------------------------------
\section{Polynomials in Fields}
\begin{align*}
    p(x) \in K(x)& \\ 
    & p = f\cdot g, f, g \in K(x) \\ 
    &\quad \quad \quad  \quad  \partial f, \partial g > 0  \\    
    &\quad \quad\quad  \quad  \partial f, \partial g < \partial p \\    
    &\text{ case 1 }\partial p = 2\\ 
    &\quad\quad  p = f \cdot g \\ 
    & \quad\quad \partial f = \partial g = 1 \\
    &\quad\quad\text{p is irreduciable } \leftrightarrow \text{p doesn't have units in K (quadrati formula)} \\ 
    & \text{case 2:} \partial p = 3 \\ 
    & \quad \quad\partial f = 1, \\ 
    & \quad \quad\partial g = 2  \\ 
    &\quad \quad\text{p is irrudicible } \leftrightarrow \text{ p doens't have a root in K} \\ 
    &\text{case 3:} \partial p = 4 \\ 
    &\quad\quad \partial f = 2, \partial g = 2 \text{ or } \partial f = 1, \partial g = 3
\end{align*}
\begin{align*}
    &\mathbb{Q}[x] \\ 
    &p \in \mathbb{Z}[x] ?
\end{align*}
\begin{lemma}[Gauss' Lemma] \leavevmode \\ 
    $h\in \mathbb{Z}[x] \text{ irriducible } \Rightarrow \text{ h is irreducible in }\mathbb{Q}[x]$ \\ 
    ($\Leftarrow$): is this true? (no)\\
    \quad \begin{align*}
        &h = f \cdot g \\ 
        &h = 2x+ 2 = 2(x+1) \text{ where } \partial f, \partial g < \partial h
    \end{align*}
    this constant is not a unit. 
    \begin{proof}
        \leavevmode \\ 
        Suppose $h = f\cdot g\text{ where }f, g \in \mathbb{Q}[x]$.  Clear denominators in $f, g$. \\ 
        There is the smallest positive integer $k$ such that $k \cdot h = \bar{f}\cdot \bar{g}$ where $\bar{f}, \bar{g} \in \mathbb{Z}[x]$ \\ 
        There is a prime $p$ dividing $k$. Let's look at $kh = \bar{f} \bar{g}$ in $\mathbb{Z}_p(x)$\\ 
        \quad in $\mathbb{Z}_p, 0 = \bar{f}_p \cdot \bar{g}_p$ Z: integral domain, so either one must be 0, \\  \\ 
        $\bar{f_p} = 0 $ or $\bar{g_p} = 0 \rightarrow $ either all coefficients of $\bar{f}$ or all coefficients of $\bar{g}$ iare  divisible by $p$ $\rightarrow $ k can be reduced. contradiction. \\ 
    \end{proof}
\end{lemma}
\section{Eisenstein's Criterion}
\begin{align*}
&h \in \mathbb{Z}[x] \\
&h = a_0 + a_1 x + \dots + a_nx^n
\end{align*}
suppose that there exists a prime $p$ such that:
\begin{enumerate}
    \item $p | a_0, \dots, a_{n-1}$
    \item $p \not | a_n$
    \item $p^2 \not | a_0$
\end{enumerate}
$\rightarrow f$ is irreducible in $\mathbb{Q}[x]$
\begin{proof}
    \leavevmode \\ 
    suffice to show that $h$ is irridubcible in $\mathbb{Z}[x]$ (Gauss lemma) \\ 
    Suppose $h = f \cdot g,\text{ where }f,g \in \mathbb{Z} [x]$ and $\partial f, \partial g < \partial h$ \\ 
    Let's look at $h  = fg \mod p$ \\ 
    $h_p = f_p g_p$ \\ 
    $a_nx^n= f_p g_p$  \\ 
    $a_n \not \equiv 0 \mod p$ \\ 
    look $a_0, p \mid a_o, p^2\nmid a_0$
    \\ 
    $\rightarrow p \text{ divides constant term} g, f \text{ or } g \text{ but not both}$ \\ \\ 
    WLOG,  \\ 
    $p \mid \text{ constant term of }g$ and $p\nmid \text{constant term of } g \rightarrow g_p$ is a polynomial with a constant term \\ 
    \\ 
    $$a_nx^n =f_p \cdot g_p$$ 
    $\mathbb{Z}_p(x) $ UFD but we have two different factorizations.. contradiction
\end{proof}
 \end{document}
