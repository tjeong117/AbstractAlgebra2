\documentclass{article}
\usepackage{amsmath, amsthm, amssymb, amsfonts}
\usepackage{thmtools}
\usepackage{graphicx}
\usepackage{setspace}
\usepackage{geometry}
\usepackage{float}
\usepackage{hyperref}
\usepackage[utf8]{inputenc}
\usepackage[english]{babel}
\usepackage{framed}
\usepackage[dvipsnames]{xcolor}
\usepackage{tcolorbox}
\usepackage{amsmath}
\usepackage{array}
\usepackage{tikz} 
\usepackage{multirow}
\usepackage{tcolorbox}
\usepackage{xcolor}

% Define a new color for the example box



\colorlet{LightGray}{White!90!Periwinkle}
\colorlet{LightOrange}{Orange!15}
\colorlet{LightGreen}{Green!15}

\newcommand{\HRule}[1]{\rule{\linewidth}{#1}}

\colorlet{LightGray}{black!10}
\colorlet{LightOrange}{orange!15}
\colorlet{LightGreen}{green!15}
\colorlet{LightBlue}{blue!15}
\colorlet{LightCyan}{cyan!15}



\declaretheoremstyle[name=Theorem,]{thmsty}
\declaretheorem[style=thmsty,numberwithin=section]{theorem}
\usepackage{tcolorbox} % Add missing package
\tcolorboxenvironment{theorem}{colback=LightGray}

\declaretheoremstyle[name=Definition,]{thmsty}
\declaretheorem[style=thmsty,numberwithin=section]{definition}
\tcolorboxenvironment{definition}{colback=LightBlue}

\declaretheoremstyle[name=Proposition,]{prosty}
\declaretheorem[style=prosty,numberlike=theorem]{proposition}
\tcolorboxenvironment{proposition}{colback=LightOrange}


\declaretheoremstyle[name=Example,]{prosty}
\declaretheorem[style=prosty,numberlike=theorem]{example}
\tcolorboxenvironment{example}{colback=LightOrange}

\declaretheoremstyle[name=Axiom,]{prcpsty}
\declaretheorem[style=prcpsty,numberlike=theorem]{axiom}
\tcolorboxenvironment{axiom}{colback=LightGreen}

\declaretheoremstyle[name=Lemma,]{prcpsty}
\declaretheorem[style=prcpsty,numberlike=theorem]{lemma}
\tcolorboxenvironment{lemma}{colback=LightCyan}





\setstretch{1.2}
\geometry{
    textheight=9in,
    textwidth=5.5in,
    top=1in,
    headheight=12pt,
    headsep=25pt,
    footskip=30pt
}

% ------------------------------------------------------------------------------

\begin{document}

% ------------------------------------------------------------------------------
% Cover Page and ToC
% ------------------------------------------------------------------------------

\title{ \normalsize \textsc{}
		\\ [2.0cm]
		\HRule{1.5pt} \\
		\LARGE \textbf{\uppercase{Lecture 1}}
		\HRule{2.0pt} \\ [0.6cm] \LARGE{}
		}

\date{\today}
\author{\textbf{Author} \\ 
		Tom Jeong
        }

\maketitle

\tableofcontents
\newpage

% ------------------------------------------------------------------------------
\section{Rings}
$\mathbb{Z}$, +: addition, $\cdot$ multiplication \\ 
\begin{align}
    &a+ (b+c) = (a+b) + c \text{ 0 - aditive identity} \\ 
    &(ab)c = a(bc) \\ 
    & a + b = b + a \text{  : multiplication doesn't have to be communtative} \\ 
    &(a +b)c = ac + bc \text{ : distributinos} \\ 
    &a + (-a) = 0 \text{ : additive inverse}
 \end{align}
$1_R$- Multiplicative idenfity (if exists) \\ if there exists $1_R \in R$ then $R$ is called unital or ring with unity. \\ 
If $ab= ba, \forall a,b \in R$ then $R$ is a commutative ring  \\  \\ 
$(R, +)$ is an abeligan group

\underline{examples}: $(\mathbb{Z}, + , \cdot), (2\mathbb{Z}, +, \cdot)$ ring without 1, \\ $M_n(R) : n \times n $ matrices with entreis in a ring R. - Not communatitive ring.
\\ 
\noindent\rule{\textwidth}{0.5pt}

\section{Integral Domain}
Ring $D$ (commutative, unital) is an Integral Domain if it enjoys cancellation property: $$ab= ac \rightarrow b = c \text{ } (\text{if } a \not = 0)$$
\begin{definition}[Equivalent]
    \leavevmode \\ 
    $ab = ac \longleftrightarrow ab = ac = 0 \longleftrightarrow a(b-c) =0$ \\ 
    in other words: $$ab = 0 \rightarrow (a = 0) || (b = 0)$$
\end{definition}
\begin{definition}[zero divisors]
    $$ab= 0 \text{ and } a,b \not = 0 $$ then 
    a and b are zero divisors
\end{definition}
$\mathbb{Z}/6\mathbb{Z}$ (integer mod 6) or $\mathbb{Z}_6$ (same thing diff notation)
\\ proving that this not an I.D. : 
$2 \cdot 3 = 0$ but $2, 3 \not = 0$ \\ 
\noindent\rule{\textwidth}{0.5pt}

\section{Fields}
A commutative ring where every element has a multiplicative inverse is called a Field.
\begin{definition}[unit]
    \leavevmode \\ 
    An element $a \in R$ is called a unit if it has a multiplicative inverse

\end{definition}
(groups of unit) $(ab)^{-1}= b^{-1}a^{-1}$: units in a ring forms a a group. Commutative ring then it is an abelian group.  \\ 
\underbar{example} $\mathbb{Z}$: units are $\{-1, 1\} \cong \mathbb{Z}_2$


K-field, $K^{\star} = K \backslash \{0\}$ and $(K^{\star}, \cdot)$ is an abelian group  \\ \\ 
\underbar{example}: $R = \{a + b\sqrt{2} | a,b \in \mathbb{Z}\}$ 

I claim that $(\sqrt{2} + 1)$ is a unit. we show that it has a mult. inverse: $(\sqrt{2} + 1)(\sqrt{2} - 1) = 1$ and we see that $(\sqrt{2} + 1)^k$ are all distinct units for $k \in \mathbb{Z}, k \geq 0$\\
\noindent\rule{\textwidth}{0.5pt}
\section{Ring Homorphisms}
maps between rings: (respect the structure of addition, multiplication)
$\phi: R \to S$ is a homomorphism if: 

\begin{enumerate}
    \item $\phi(a+b) = \phi(a) + \phi(b) $
    \item $\phi(ab) = \phi(a)\phi(b)$
\end{enumerate}
\underline{property}:\begin{enumerate}
    \item $\phi(O_R) = O_S$ 
    \item $\phi(-r) = -\phi(r)$ 
    \item $\phi(R) \leq S$ (subring of S)
\end{enumerate}
\begin{definition}[ker]
    
    $$ker(\phi) = \{r \in R| \phi(r) = 0_S\}$$

\end{definition}
\begin{proposition}
    $ker\phi$ is a subring of R
\end{proposition}
\begin{proof}
    \leavevmode \\ 
    $\phi(ab) = \phi(a) \phi(b) = 0 $ \\ 
    $a, b \in ker\phi$   \\ 
    $\phi(a + b) = \phi(a) + \phi(b ) = 0$ \\ 
    $\phi(ra) = 0 $ for all $r \ in R$ 
\end{proof}
\begin{definition}[ideal] \leavevmode \\ 
    $ker\phi$ is an ideal of R. \\ 
    Ideal: subring closed under multiplcation by any element of R. \\ 
    ideal $I$ of $R$ if $\forall a, b \in I, r \in R \rightarrow a,b\in I$
    
\end{definition}

\begin{enumerate}
    \item closure under addition and additive inverse $a-b \in I$ 
    \item $ra \in I $
\end{enumerate}

\underline{Example}: $2\mathbb{Z}$ is an ideal in $\mathbb{Z}$ \\ 
k-field: $\{0\}, R$ "not interesting ideals  contains only zero
\\ 
Proper Ideal of $R$ is an "interesting" ideal 
A proper ideal is any ideal that is a strict subset of $R$ (so not $R$ itself). These are considered ``interesting'' because they:

\begin{itemize}
   \item Reveal the ring's algebraic structure
   \item Help classify rings  
   \item Are used to construct quotient rings
   \item Can determine properties like primality and maximality
\end{itemize}

For example, in $\mathbb{Z}$ (integers), $(4) = \{\ldots,-8,-4,0,4,8,\ldots\}$ is an interesting proper ideal, while $\{0\}$ and $\mathbb{Z}$ are uninteresting.


:: $I$- ideal.
$$I \subsetneq R \leftrightarrows 1_R \not \in I \leftrightarrows I \text{ contains no units}$$
\begin{proof}
    Let $I \subset K$ be a proper ideal. If $a \neq 0$ and $a \in I$, then $a \cdot a^{-1} \in I$ since $I$ is an ideal. 
    But $a \cdot a^{-1} = 1$ (multiplicative identity), therefore $1 \in I$.
    Since $I$ is an ideal, for any $k \in K$, $k \cdot 1 = k \in I$.
    Thus $I = K$, contradicting that $I$ is proper.
    \end{proof}
    \noindent\rule{\textwidth}{0.5pt}
\section{Quotient Rings}
$R / I $: cosets of I : 
\begin{align*}
    & a + I \forall a \in R \\ 
    & (a + I) \cdot (b + I) = ab + I \\ 
    & (a +I) + (b + I) = a + b + I   \\ 
    & 0_{R/I}  = 0 + I = I 
\end{align*}

\begin{definition}[canonical projection]
    \begin{align*}
        &\pi: R \to R / I \\ 
        &a \mapsto a + I \\ 
        &ker\pi = I \\ 
        &\phi: R \to S \\ 
        &im\phi \cong R / \ker\phi
     \end{align*}
    
\end{definition}
All ring homomorphisms are canonical projections (because kernel is always the ideal.)
\noindent\rule{\textwidth}{0.5pt}
\section{Fields of Fraction (tbc)}
$\mathbb{Z} \to \mathbb{Q}$, \\ D integral Domain 

\begin{definition}[rational]
     \leavevmode \\ 
     $(a,b): \frac{a}{b},  a, b \in \mathbb{Z} $
     \\ but we have a problem: (1,2) = (2,4) = (3,6) = ... 
     So tequivalence classes of pairs of integersL 
     $$(a,b) \sim (a', b') \leftrightarrows ab' = a'b$$ 
\end{definition}
actions on rational numbers
\begin{enumerate}
    \item multiplication: $(a,b) \cdot (a', b' ) = (aa', bb')$ ID is needed. 
    \item addition: $(a,b) + (a', b') = \frac{a}{b} + \frac{a'}{b'} =\frac{ab' + a'b}{bb'} = (ab' + a'b, bb')$
\end{enumerate}
Why ID is needed: 
\begin{proof}
    The identity element (1,1) is required because:
    
    \begin{enumerate}
       \item In a ring, multiplication must have an identity element
       \item For component-wise multiplication $(a,b) \cdot (1,1) = (a \cdot 1, b \cdot 1) = (a,b)$ must hold
       \item Without $(1,1)$, the structure would not satisfy ring axioms:
       \begin{itemize}
           \item Existence of multiplicative identity
           \item Distributive property over addition 
           \item Closure under multiplication
       \end{itemize}
    \end{enumerate}
    
    This ensures the direct product maintains ring properties from its component rings.
    \end{proof}


equivalence relation? 
\begin{enumerate}
    \item $(a,b) \sim (a,b) $ 
    \item $(a,b) \sim (a', b') \leftrightarrow (a', b') \sim (a,b)$ 
    \item $(a,b) \sim (a', b') $ and $(a',b') \sim (a'', b'') \rightarrow (a,b) \sim (a'', b'')$
\end{enumerate}
If $(a,b) \sim (a',b')$ then $\exists k_1 \in \mathbb{Q}: a-a'=k_1$ and $b-b'=k_1$ \\
If $(a',b') \sim (a'',b'')$ then $\exists k_2 \in \mathbb{Q}: a'-a''=k_2$ and $b'-b''=k_2$ \\
Adding equations: $(a-a')+(a'-a'')=k_1+k_2$ and $(b-b')+(b'-b'')=k_1+k_2$ \\
Therefore $a-a''=k_1+k_2$ and $b-b''=k_1+k_2$ where $k_1+k_2 \in \mathbb{Q}$ \\
Thus $(a,b) \sim (a'',b'')$ 
( there is an easier way i just pasted the above from claude )

 
this construction adds a multiplicative inverse to every non-zero element, making it into a field. $Q(D)$



\end{document}
