\documentclass{article}

% Basic Math Packages
\usepackage{amsmath, amsthm, amssymb, amsfonts}
\usepackage{cancel}

% Layout and Design
\usepackage{geometry}
\usepackage{setspace}
\usepackage{float}
\usepackage{graphicx}
\usepackage{array}
\usepackage{multirow}
\usepackage{framed}

% Colors and Boxes
\usepackage[dvipsnames]{xcolor}
\usepackage{tcolorbox}
\usepackage{tikz}
\usepackage{tikz-cd}

% Document Properties
\usepackage[utf8]{inputenc}
\usepackage[english]{babel}
\usepackage{hyperref}  % Load last
\usepackage{thmtools}
\usepackage{wasysym}

% Custom Colors
\colorlet{LightGray}{White!90!Periwinkle}
\colorlet{LightOrange}{Orange!15}
\colorlet{LightGreen}{Green!15}
\colorlet{LightBlue}{blue!15}
\colorlet{LightCyan}{cyan!15}

% Custom Commands
\newcommand{\HRule}[1]{\rule{\linewidth}{#1}}
\newcommand\Tau{\mathcal{T}}% Caligraphic T for example
% Theorem Environments
\declaretheoremstyle[name=Theorem]{thmsty}
\declaretheorem[style=thmsty,numberwithin=section]{theorem}
\tcolorboxenvironment{theorem}{colback=LightGray}

\declaretheoremstyle[name=Definition]{thmsty}
\declaretheorem[style=thmsty,numberwithin=section]{definition}
\tcolorboxenvironment{definition}{colback=LightBlue}

\declaretheoremstyle[name=Proposition]{prosty}
\declaretheorem[style=prosty,numberlike=theorem]{proposition}
\tcolorboxenvironment{proposition}{colback=LightOrange}

\declaretheoremstyle[name=Corollary]{prosty}
\declaretheorem[style=prosty,numberlike=theorem]{corollary}
\tcolorboxenvironment{corollary}{colback=LightOrange}

\declaretheoremstyle[name=Axiom]{prcpsty}
\declaretheorem[style=prcpsty,numberlike=theorem]{axiom}
\tcolorboxenvironment{axiom}{colback=LightGreen}

\declaretheoremstyle[name=Lemma]{prcpsty}
\declaretheorem[style=prcpsty,numberlike=theorem]{lemma}
\tcolorboxenvironment{lemma}{colback=LightCyan}

% Page Layout
\setstretch{1.2}
\geometry{
    textheight=9in,
    textwidth=5.5in,
    top=1in,
    headheight=12pt,
    headsep=25pt,
    footskip=30pt
}

\begin{document}

% Title Page
\title{
    \normalsize
    \vspace{2.0cm}
    \HRule{1.5pt} \\[0.4cm]
    \LARGE \textbf{Lecture 10}
    \HRule{2.0pt} \\[0.6cm]
}
\author{
    \textbf{Author} \\
    Tom Jeong
}
\date{\today}

\maketitle
\tableofcontents
\newpage

\section{Galois Extension}
\begin{theorem}
    

$[L: K] < \infty$ Let $G = Aut(L, K)$ then $|G| \leq [L: K]$ and the following are equivalent:\begin{enumerate}
    \item $|G| = [L: K]$
    \item There exists a polynomial $f(x) \in K[x]$ such that $L$ is a splitting field of $f(x)$ and $f(x)$ has distinct roots in $L$ 
    \item $K = \{x : \sigma(x) = x \forall \sigma \in G\}$
    
\end{enumerate}
If any 1,2,3 hold then $L/K$ is called a \textit{Galois Extension} $G= Aut(L, K)$ is called the \textit{Galois Group} of $L/K$
\end{theorem}
\hrulefill \\
$f(x) = q_1^\alpha(x) \cdots q_m^\alpha(x)$ where $q_i$ are irreducible and distinct in $K[x]$ and $\alpha \geq 1$ \\
    $\bar{f}(x) = q_1(x) \cdots q_m(x)$ where $q_i$ are distinct in $L[x]$ \\
    It may happen that $q_i$ even if it's irreducible, $q_1$ has multiple roots in $L$\\
    A polynomial if $f \in K[x]$ is caled \underbar{separable} if $f$ has distinct roots in its splitting field. \\
    example: \\  
    $x^2 + 1$ doesn't have any roots in $\mathbb{Q}$.. where does the roots leave? the smallest field that contains the roots of $x^2 + 1$ is $\mathbb{Q}(i)$; inside this field we will have the roots of $x^2 + 1$ \\ \\ 
    A field $K$ is called \underbar{perfect} if all irreducible polynomials in $K[x]$ are separable. \\
    $\mathbb{Q}-$perfect field
\begin{lemma} \leavevmode \\ 
    $L$ is not  the union of finitely many proper subfields $M$, $K \subseteq M \subsetneq L$ 
    
\end{lemma}
\begin{proof}
    \leavevmode \\ 
    $K-$infinite $L-$finite dimensional $K-$vector space $dim(L) = [L:K], dim(M) < dim(L)$\\ a finite dimensional vector space is not a union of finitely many proper subspaces.  \\ 
    $K-$ finite field and $L-$finite field. \\ 
    $|L| = p^k$ \\ 
    any subfield $M$ of $L$ has $char(p) \rightarrow |M| = p^k$.. $k< n$ \\ 
    For every $k$ there is at most 1 subfield of $L$ of this size. 
    Since any subfield of $L$ of size $p^k$ is the splitting field of $x^{p^k} - x$ over $\mathbb{Z}_p$\\
    $1+ p+p^2 + \cdots + p^{k-1} <  p^n$ since $1 + p + \cdots + p^{n -1} = \cfrac{p^n - 1}{p-1} < p^n$ \\
\end{proof}
\begin{corollary} \leavevmode \\ 
    There exists $z \in L$ such that the $stab(z) = \{\sigma \in G: \sigma(z) = z\} = \{e_G\}$\\ 
    $\Rightarrow |\{\sigma(z): \sigma \in G\} | = |G|$
    \\  \\ 
    $|G| = n$ and we have that $G = \{\sigma_1, \sigma_2, \cdots, \sigma_n\}$ \\ Orbit of $z$ $\sigma_1(z), \sigma_2(z), \cdots, \sigma_n(z)$ \\ 
    We know that these are distinct elements of $L$ \\
    
    $K \subsetneqq K(z) \subset L$ \\ 
    $z$ has minimal polyonomial $f_z$\\ 
    $$[L:K] \geq [K(z) :K] = deg(f_z) \geq n = |G|$$
\end{corollary}
\begin{proof}
    \leavevmode \\ 
    For 
    $\sigma \in G$, $M_{\sigma} = \{x \in L: \sigma(x) = x\}$ \\ 
    $M_{\sigma}$ is a field, $K \subseteq M$ \\
    $a,b \in M_{\sigma}$, $\sigma(a + b) = \sigma(a) + \sigma(b) = a + b$ \\
    $\sigma(ab) = \sigma(a) \sigma(b) = ab$ \\
    $\sigma(-a) = - \sigma(a)$ \\ 
    For every $\sigma \in G$ we are prohibiting a subfield $M_{\sigma}$ since $L$ is not the union of finitely many propert subfields \\ such $z$ exists. There exists $z\in L \backslash \bigcup_{\sigma \in G | \sigma \not = e} M_{\sigma}$ \\
\end{proof}
\begin{proof}
    \leavevmode \\ 
    \begin{enumerate}
        \item (1) $\Rightarrow$ (2) \\ 
        by the corollary we estabilished 
        $$[L:K] \geq [K(z) :K] = deg(f_z) \geq n = |G|$$
        $deg(f_z) = n \rightarrow \sigma_1, \dots \sigma_n$ are all of the roots of $f_z \rightarrow f_z$ has distinct roots in $L$ \\ 
        $K(z) = L \rightarrow L$ is the splitting field of $f_z$ since $f_z $ splits over $L$ \\ and $K(z) = L \rightarrow f_z$ does not split over any subfield. 
        
    \end{enumerate}
  
    
    
\end{proof}
\end{document}