\documentclass{article}
\usepackage{amsmath, amsthm, amssymb, amsfonts}
\usepackage{thmtools}
\usepackage{graphicx}
\usepackage{setspace}
\usepackage{cancel}
\usepackage{geometry}
\usepackage{float}
\usepackage{hyperref}
\usepackage[utf8]{inputenc}
\usepackage[english]{babel}
\usepackage{framed}
\usepackage[dvipsnames]{xcolor}
\usepackage{tcolorbox}
\usepackage{amsmath}
\usepackage{array}
\usepackage{tikz} 
\usepackage{multirow}
\usepackage{tcolorbox}
\usepackage{xcolor}
\usepackage{tikz-cd}
\usepackage{xcolor}
\usepackage{wasysym}

% Define a new color for the example box



\colorlet{LightGray}{White!90!Periwinkle}
\colorlet{LightOrange}{Orange!15}
\colorlet{LightGreen}{Green!15}

\newcommand{\HRule}[1]{\rule{\linewidth}{#1}}

\colorlet{LightGray}{black!10}
\colorlet{LightOrange}{orange!15}
\colorlet{LightGreen}{green!15}
\colorlet{LightBlue}{blue!15}
\colorlet{LightCyan}{cyan!15}



\declaretheoremstyle[name=Theorem,]{thmsty}
\declaretheorem[style=thmsty,numberwithin=section]{theorem}
\usepackage{tcolorbox} % Add missing package
\tcolorboxenvironment{theorem}{colback=LightGray}

\declaretheoremstyle[name=Definition,]{thmsty}
\declaretheorem[style=thmsty,numberwithin=section]{definition}
\tcolorboxenvironment{definition}{colback=LightBlue}

\declaretheoremstyle[name=Proposition,]{prosty}
\declaretheorem[style=prosty,numberlike=theorem]{proposition}
\tcolorboxenvironment{proposition}{colback=LightOrange}


\declaretheoremstyle[name=Example,]{prosty}
\declaretheorem[style=prosty,numberlike=theorem]{example}
\tcolorboxenvironment{example}{colback=LightOrange}

\declaretheoremstyle[name=Axiom,]{prcpsty}
\declaretheorem[style=prcpsty,numberlike=theorem]{axiom}
\tcolorboxenvironment{axiom}{colback=LightGreen}

\declaretheoremstyle[name=Lemma,]{prcpsty}
\declaretheorem[style=prcpsty,numberlike=theorem]{lemma}
\tcolorboxenvironment{lemma}{colback=LightCyan}





\setstretch{1.2}
\geometry{
    textheight=9in,
    textwidth=5.5in,
    top=1in,
    headheight=12pt,
    headsep=25pt,
    footskip=30pt
}

% ------------------------------------------------------------------------------

\begin{document}

% ------------------------------------------------------------------------------
% Cover Page and ToC
% ------------------------------------------------------------------------------

\title{ \normalsize \textsc{}
		\\ [2.0cm]
		\HRule{1.5pt} \\
		\LARGE \textbf{\uppercase{Lecture 5}}
		\HRule{2.0pt} \\ [0.6cm] \LARGE{}
		}

\date{\today}
\author{\textbf{Author} \\ 
		Tom Jeong
        }

\maketitle

\tableofcontents
\newpage

% ------------------------------------------------------------------------------
\section{Degree of polynomials in fields}
$K[x] / f(x)$ where f(x) is irreducible. What is $[L:K]$? where $d= \delta f(x)$
\begin{align*}
    &\bar{1}, \bar{x}, \bar{x}^2, \ldots, \bar{x}^{d-1} \quad \text{basis for } L \\
    &\bar{1} = 1 + <f> \\ 
    &\bar{x} = x + <f> \text{ spn L over K} \\
    &k(x) / <f> \text{ is isomorphic to } L \\
    &\alpha_0 + \alpha_1x + \ldots + \alpha_{d-1}x^{d-1} \in L \text{ where } \alpha_i \in K \\
    &\alpha_0 + \alpha_1 x + \dots \alpha_{d-1}x^{d-1}\in <f> \text{ which is a contradiction the only way this can happen } \\ 
    &\rightarrow \alpha_0 = \alpha_1 = \ldots = \alpha_{d-1} = 0 \\ 
    &\text{ so } \bar{1}, \bar{x}, \ldots, \bar{x}^{d-1} \text{ are linearly independent} \\
    &\text{ so } \text{dim}_K L = d \text{ so } [L:K] = d
\end{align*}
we take $L / K$ where $\alpha \in L$ smallest field containing K and L.\\ 
$K[\alpha] = \text {polynomials in alpha with coefficients in K}$ \\ 
\begin{align*}
    &c_0 + c_1 \alpha + c_2 \alpha^2 + \ldots + c_{d-1}\alpha^{d-1} \quad \text{where } c_i \in K \\
    &K[\alpha] \subseteq L \text{ subring of L} \\ 
    &K(\alpha) = \text { rationals functions in } \alpha = \{\cfrac{p(\alpha)}{q(\alpha) }: q(\alpha \not = 0, p,q \in K[x])\} 
\end{align*}

$K(\alpha)$ is the smallest subfield of L \\ 
Smallest subfield of L containing K and $\alpha$ \\
\begin{theorem}
    $L / K, \alpha \in L$ then \begin{enumerate}
        \item If there is no polynomial in K[x] s.t. $p(\alpha) = 0$ then $K(\alpha) \cong K(x)$. 
        \item if there is a polynomial in $K[x]$ such that $p(\alpha) = 0$ then $K(\alpha) = K[\alpha] \equiv \langle m(x) \rangle $
    \end{enumerate}
    where $m(x)$ is m-irreducible polyonimla  m(d) = 0. 
 \\ 
 In case (1) $\alpha$ is called transcendental over K. 
 \\ 
 $\mathbb{R} / \mathbb{Q}, \pi, e$ is trascendental. what about $\pi + e, \pi \cdot e$? it is not easy to show that a number is transcendental
In Case(2) we say that alpha is algebraic over K.
for example $\mathbb{R } / \mathbb{Q}$ algebraic number are countable. R is uncountable 
\end{theorem}
$L / K $ , $ d\in L$ \\ 
$I(\alpha) = \{p \in K(x) : p(\alpha) = 0\}$, \\ 
$ p \in I(\alpha), p\cdot r, r \in K(x) $ \\ 
$p \cdot r(\alpha) = p(\alpha) \cdot r(\alpha) = 0$ \\ 
case 1: $I(\alpha) = 0$ \\
case 2: $I(\alpha) \not = 0$ \\ 
I is principale $I = \langle f \rangle$ \\
f makes it monic -- $m = \frac{f}{\text{lead coeff of f}}$ \\ 
this makes uniequely defind $I(\alpha) = \langle m \rangle$ \\
m - minimal polynomial of $\alpha$. 

\begin{proof}
\begin{enumerate}
    \item case 1: $I(\alpha) = 0$ \\ 
            then $K(\alpha) = K(x)$ \\
            $K(\alpha) \cong K(x)$ \\
            $\phi: K(x) \to K(\alpha)$ \\
            $\frac{p(x)}{q(x)} \to \frac{p(\alpha)}{q(\alpha)}$ \\
            $\phi$ is an onto isomorphism.\\ 
            $ker \phi = 0$? $\frac{p(x)}{q(x)} \in ker \phi$ since $p(\alpha) = 0$\\
    \item case 2: $I(\alpha) = \langle m \rangle$ \\
            $K(\alpha) = K[\alpha] = \langle m \rangle$ \\
            $K(\alpha) \cong K[x] / \langle m \rangle$ \\
            $\phi: K[x] \to K(\alpha)$ \\
            $p(x) \to p(\alpha)$ \\
            $\phi$ is onto \\
            $ker \phi = \langle m \rangle$ \\
            $p(x) \in ker \phi \implies p(\alpha) = 0 \implies p(x) = m(x) \cdot q(x)$ \\
            $p(x) \in \langle m \rangle$ \\
            $K[x] / \langle m \rangle \cong K(\alpha)$ 
\end{enumerate}
\end{proof}
$K(x) / \langle f(x) \rangle \cong K(\alpha)$ \\ where f(x) is irreducioble. 

\begin{align*}
    &x^3 - 2 \text{ is irreducible using Eisenstein p = 2}  \\
    &\text{the roots are } \sqrt[3]{2}, \omega \sqrt[3]{2}, \omega^2 \sqrt[3]{2} \text{ where } \omega = e^{\frac{2\pi i}{3}} \\ 
    &\mathbb{Q}[\sqrt[3]{2}] \cong \mathbb{Q}(x) / \langle x^3 - 2 \rangle \cong \mathbb{Q}[\omega \sqrt[3]{2}] \cong \mathbb{Q}[\omega^2 \sqrt[3]{2}] \\
\end{align*}
Smallest field obtaining all roots of f is called splitting field of f. 
\end{document}
