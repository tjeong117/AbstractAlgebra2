\documentclass{article}
\usepackage{amsmath, amsthm, amssymb, amsfonts}
\usepackage{thmtools}
\usepackage{graphicx}
\usepackage{setspace}
\usepackage{cancel}
\usepackage{geometry}
\usepackage{float}
\usepackage{hyperref}
\usepackage[utf8]{inputenc}
\usepackage[english]{babel}
\usepackage{framed}
\usepackage[dvipsnames]{xcolor}
\usepackage{tcolorbox}
\usepackage{amsmath}
\usepackage{array}
\usepackage{tikz} 
\usepackage{multirow}
\usepackage{tcolorbox}
\usepackage{xcolor}
\usepackage{tikz-cd}
\usepackage{xcolor}
\usepackage{wasysym}

% Define a new color for the example box



\colorlet{LightGray}{White!90!Periwinkle}
\colorlet{LightOrange}{Orange!15}
\colorlet{LightGreen}{Green!15}

\newcommand{\HRule}[1]{\rule{\linewidth}{#1}}

\colorlet{LightGray}{black!10}
\colorlet{LightOrange}{orange!15}
\colorlet{LightGreen}{green!15}
\colorlet{LightBlue}{blue!15}
\colorlet{LightCyan}{cyan!15}



\declaretheoremstyle[name=Theorem,]{thmsty}
\declaretheorem[style=thmsty,numberwithin=section]{theorem}
\usepackage{tcolorbox} % Add missing package
\tcolorboxenvironment{theorem}{colback=LightGray}

\declaretheoremstyle[name=Definition,]{thmsty}
\declaretheorem[style=thmsty,numberwithin=section]{definition}
\tcolorboxenvironment{definition}{colback=LightBlue}

\declaretheoremstyle[name=Proposition,]{prosty}
\declaretheorem[style=prosty,numberlike=theorem]{proposition}
\tcolorboxenvironment{proposition}{colback=LightOrange}


\declaretheoremstyle[name=Example,]{prosty}
\declaretheorem[style=prosty,numberlike=theorem]{example}
\tcolorboxenvironment{example}{colback=LightOrange}

\declaretheoremstyle[name=Axiom,]{prcpsty}
\declaretheorem[style=prcpsty,numberlike=theorem]{axiom}
\tcolorboxenvironment{axiom}{colback=LightGreen}

\declaretheoremstyle[name=Lemma,]{prcpsty}
\declaretheorem[style=prcpsty,numberlike=theorem]{lemma}
\tcolorboxenvironment{lemma}{colback=LightCyan}





\setstretch{1.2}
\geometry{
    textheight=9in,
    textwidth=5.5in,
    top=1in,
    headheight=12pt,
    headsep=25pt,
    footskip=30pt
}

% ------------------------------------------------------------------------------

\begin{document}

% ------------------------------------------------------------------------------
% Cover Page and ToC
% ------------------------------------------------------------------------------

\title{ \normalsize \textsc{}
		\\ [2.0cm]
		\HRule{1.5pt} \\
		\LARGE \textbf{\uppercase{Lecture 6}}
		\HRule{2.0pt} \\ [0.6cm] \LARGE{}
		}

\date{\today}
\author{\textbf{Author} \\ 
		Tom Jeong
        }

\maketitle

\tableofcontents
\newpage

% ------------------------------------------------------------------------------
\section{Useful Facts}
\begin{enumerate}
    \item $L/K$ , $L = K \leftrightarrows [L:K] = 1$
    \item $L/K$, $\alpha \in L$, $f(x) \in k(x), f(\alpha) = 0$ then $f$ is irreducible $\leftrightarrows [k(\alpha):k] = \partial f$
\end{enumerate}
on number 2 specifically, f is irreducible then $k(\alpha) \cong k(x)/ f(x), [k(x) / f(x) : k] = \partial f$ \\ 
$I(\alpha) = \{p \in k(x): p (\alpha) = 0\}$, $I(\alpha) = \langle m(\alpha) \rangle$ \\ 


$m(\alpha) = \text{minimal polynomial of } \alpha$  thus irreducible.  
\\ 
$[k(\alpha):k] = \partial m(\alpha)$ \\
$[k(\alpha):k] = [k(x)/m(x):k] = \partial m(x)$ \\
So f must be a constant multiple thus f is irreducible. 

\section{ajoining multiple elements}
$k(\alpha_1, \alpha_2, \ldots, \alpha_s)$ where $\alpha_i$- minimal polynomial $m_i$, $d_i = \partial m_i$ \\
3. $[k(\alpha_1, \alpha_2, \ldots, \alpha_s):k] \leq d_1 \cdot d_2 \cdot \ldots \cdot d_s$ \\

$[k(\alpha_1,\alpha_2):k]= [k(\alpha_1,\alpha_2):k(\alpha_1)] \cdot [k(\alpha_1):k] = [k(\alpha_1,\alpha_2):k(\alpha_1)] \cdot d_1$ .. tower theorem \\ 
want to show that $[k(\alpha_1,\alpha_2):k] \leq d_2$ \\
$d_2$ = degree of the minimal polynomial $m_2(x)$ of $d_2$ in $k(x)$. $m_2(x)$ is irreducible in $k(x)$ \\

$m_2(x)$ may become reducible in $k(d_1)[x]$ \\
$\bar{m_2}(x)$ is the minimal polynomial of $d_2$ in $k(d_1)[x]$ \\
we know $\partial \bar{m}_2 \leq d_2$ \\

$[k(\alpha_1, \alpha_2, \alpha_3): k ] = [k(\alpha_1, \alpha_2, \alpha_3): k(\alpha_1, \alpha_2)] \cdot [k(\alpha_1, \alpha_2): k(\alpha_1)] \cdot [k(\alpha_1): k] \leq d_3 \cdot d_2 \cdot d_1$ \\

\subsection{examlpes}
$$[\mathbb{Q}(\sqrt[3]{2}, \sqrt{2}): \mathbb{Q}]$$
before that lets see $[\mathbb{Q}(\sqrt{2}) : \mathbb{Q}]$
\begin{align*}
    &x^2 - 2 \text{ is irreducible in } \mathbb{Q}[x] \\ 
    &\text{thus, }     [\mathbb{Q}(\sqrt{2}) : \mathbb{Q}] = 2\\ 
    &[\mathbb{Q}(\sqrt[3]{2}) : \mathbb{Q}]  = 3 \\ 
    &x^3 - 2 \text{ Eisenstein criteria } \\ 
    &\text{thus, } [\mathbb{Q}(\sqrt[3]{2}) : \mathbb{Q}] = 3\\ 
    &\rightarrowtail[\mathbb{Q}(\sqrt[3]{2}, \sqrt{2}): \mathbb{Q}] \leq 6 \text{ now trying to pvoe that it is equal to 6}\\ 
    &\text{tower theorem }[\mathbb{Q}(\sqrt[3]{2}, \sqrt{2}): \mathbb{Q}] \text{ is divisible by 2 and 3}  \\ 
    &\rightarrow [\mathbb{Q}(\sqrt[3]{2}, \sqrt{2}): \mathbb{Q}] = 6
\end{align*} 

$$[\mathbb{Q}(\sqrt{2} + \sqrt{3}) : \mathbb{Q}]$$
\begin{align*}
    &\mathbb{Q}(\sqrt{2}, \sqrt{3}) \geq \mathbb{Q}(\sqrt{2} + \sqrt{3}) \text { we show that it is actually equal } \\
    &\cfrac{1}{\sqrt{2} + \sqrt{3}}   = \cfrac{\sqrt{3} - \sqrt{2}}{(\sqrt{2} + \sqrt{3})(\sqrt{3} - \sqrt{2})} = \cfrac{\sqrt{3} - \sqrt{2}}{3 - 2} = \sqrt{3} - \sqrt{2} \in \mathbb{Q}(\sqrt{2}, \sqrt{3}) \\
    &\mathbb{Q}(\sqrt{2} + \sqrt{3}) = \mathbb{Q}(\sqrt{2}, \sqrt{3})   \\
    &[\mathbb{Q}(\sqrt{2}, \sqrt{3}) : \mathbb{Q}] = \{2, 4\} \\ 
    &\text{if the answer was 2, } [\mathbb{Q}(\sqrt{3}, \sqrt{2}): \mathbb{Q}(\sqrt{2})] = 1 \text{ making the fields the equal} 
\end{align*}

Show that $x^3 - 2$ is irreducible over $\mathbb{Q}[i]$ \\
$x^3 - 2$ is irreducible over $\mathbb{Q}[i] \leftrightarrows  [Q[\sqrt[3]{2}] : \mathbb{Q} [i]] = 3$ 


\section{squaring a circle}
\end{document}
