\documentclass{article}

% Basic Math Packages
\usepackage{amsmath, amsthm, amssymb, amsfonts}
\usepackage{cancel}

% Layout and Design
\usepackage{geometry}
\usepackage{setspace}
\usepackage{float}
\usepackage{graphicx}
\usepackage{array}
\usepackage{multirow}
\usepackage{framed}

% Colors and Boxes
\usepackage[dvipsnames]{xcolor}
\usepackage{tcolorbox}
\usepackage{tikz}
\usepackage{tikz-cd}

% Document Properties
\usepackage[utf8]{inputenc}
\usepackage[english]{babel}
\usepackage{hyperref}  % Load last
\usepackage{thmtools}
\usepackage{wasysym}

% Custom Colors
\colorlet{LightGray}{White!90!Periwinkle}
\colorlet{LightOrange}{Orange!15}
\colorlet{LightGreen}{Green!15}
\colorlet{LightBlue}{blue!15}
\colorlet{LightCyan}{cyan!15}

% Custom Commands
\newcommand{\HRule}[1]{\rule{\linewidth}{#1}}
\newcommand\Tau{\mathcal{T}}% Caligraphic T for example
% Theorem Environments
\declaretheoremstyle[name=Theorem]{thmsty}
\declaretheorem[style=thmsty,numberwithin=section]{theorem}
\tcolorboxenvironment{theorem}{colback=LightGray}

\declaretheoremstyle[name=Definition]{thmsty}
\declaretheorem[style=thmsty,numberwithin=section]{definition}
\tcolorboxenvironment{definition}{colback=LightBlue}

\declaretheoremstyle[name=Proposition]{prosty}
\declaretheorem[style=prosty,numberlike=theorem]{proposition}
\tcolorboxenvironment{proposition}{colback=LightOrange}

\declaretheoremstyle[name=Example]{prosty}
\declaretheorem[style=prosty,numberlike=theorem]{example}
\tcolorboxenvironment{example}{colback=LightOrange}

\declaretheoremstyle[name=Axiom]{prcpsty}
\declaretheorem[style=prcpsty,numberlike=theorem]{axiom}
\tcolorboxenvironment{axiom}{colback=LightGreen}

\declaretheoremstyle[name=Lemma]{prcpsty}
\declaretheorem[style=prcpsty,numberlike=theorem]{lemma}
\tcolorboxenvironment{lemma}{colback=LightCyan}

% Page Layout
\setstretch{1.2}
\geometry{
    textheight=9in,
    textwidth=5.5in,
    top=1in,
    headheight=12pt,
    headsep=25pt,
    footskip=30pt
}

\begin{document}

% Title Page
\title{
    \normalsize
    \vspace{2.0cm}
    \HRule{1.5pt} \\[0.4cm]
    \LARGE \textbf{Lecture 10}
    \HRule{2.0pt} \\[0.6cm]
}
\author{
    \textbf{Author} \\
    Tom Jeong
}
\date{\today}

\maketitle
\tableofcontents
\newpage

\section{house keeping}
\begin{itemize}
    \item exam next monday
    \item no materials from this lecture
\end{itemize}
% Main Content
\section{From Previous Lecture}
Discussion of $\mathbb{Q}(\sqrt[4]{2})$:
\begin{itemize}
    \item $\sqrt[4]{2}$ is a root of $x^4 - 2$
    \item Consider $-\sqrt[4]{2}$
\end{itemize}

Interesting maps $\mathbb{Q}(\sqrt[4]{2}) \to \mathbb{Q}(\sqrt[4]{2})$ that fix $\mathbb{Q}$:
\begin{enumerate}
    \item $\sqrt[4]{2} \to \sqrt[4]{2}$ (identity homomorphism)
    \item $\sqrt[4]{2} \to -\sqrt[4]{2}$
\end{enumerate}

Intermediate subfields:
$$\mathbb{Q} \subseteq \mathbb{Q}(\sqrt{2}) \subseteq \mathbb{Q}(\sqrt[4]{2})$$

\subsection{L/K Field Extension}
$Aut(L:K)$ or $Aut(L/K)$ or $Aut(L,K)$ = Automorphisms of $L$ fixing $K$

\begin{definition}[In case you forgot]
    Automorphisms are isomorphisms to itself:
    $$Aut(L:K) = \{\sigma: L \to L: \sigma \text{ is an isomorphism} \text{ and } \sigma(x) = x \text{ for all } x \in K \}$$
\end{definition}

Note: $Aut(L,K)$ is a group under composition (the book calls this the Galois group of $L/K$).

In our example: $$Aut(\mathbb{Q}(\sqrt[4]{2}), \mathbb{Q}) = \mathbb{Z}_2$$

\hrulefill

Working with finite degrees $[L:K] < \infty$:
\begin{itemize}
    \item $\alpha \in L$ has a minimal polynomial $f_\alpha \in K[x]$
    \item $f_\alpha$ has same roots in $L$
    \item For $\sigma \in Aut(K,L)$, $\sigma$ permutes the roots of $f_\alpha \in L$
\end{itemize}
$$f_\alpha(r) = 0 \Rightarrow f_\alpha(\sigma(r)) = 0$$

\hrulefill \\ 
For $f(x) \in K[x]$ and $f(r) = 0$, then
\begin{align*}
    f(\sigma(r)) &= 0 \text{ for all } \sigma \in Aut(L:K) \\
    f(x) &= a_0 + a_1 x + \dots + a_n x^n \\
    f(r) &= a_0 + a_1 r + \dots + a_n r^n = 0 \\
    f(\sigma(r)) &= a_0 + a_1 \sigma(r) + \dots + a_n \sigma(r)^n = 0 \\
    &= a_0 + a_1 \sigma(r) + \dots + a_n \sigma(r)^n = 0 \quad (\text{by the definition of } \sigma)\\
    &= \sigma(a_0) + \sigma(a_1) \sigma(r) + \sigma(a_2) \sigma(r^2)+ \dots + \sigma(a_n) \sigma(r^n) = 0 \\
    &= \sigma(a_0 + a_1 r + \dots + a_n r^n) = \sigma(0) = 0
\end{align*}
Since we have that $\sigma(r^k) = [\sigma(r)]^k$ because $\sigma(r^2) = \sigma(r) \cdot \sigma(r)$ 

Question: \begin{enumerate}
    \item How big can this group be ?
    \item For now we will show that $|Aut(L:K)|$ is finite if $[L : K] < \infty$
\end{enumerate}
\begin{theorem}
    If $[L:K] < \infty$, then $|Aut(L:K)| < \infty$
    \begin{proof}
        \leavevmode \\ 
        \begin{align*}
            &L = K(\alpha_1, \dots, \alpha_n) \\ 
            &\sigma \in Aut(L:K)  \text{ has minimal polynomial } f_{\alpha_i}\\
            &\forall \alpha \in Aut(L:K)  \text {is sepcific if we know } \sigma(\alpha_1), \dots, \sigma(\alpha_n) \\ 
            &\sigma \text{ permutes the roots of } f_{\alpha_i} \text{ in } L \\
            &\Rightarrow \forall \alpha_i \text{ there are infiniitely many possiblities for } \sigma(\alpha_i) \\
            &\Rightarrow \text{finitely many possibilities for } \sigma \in Aut(L:K) \\
            &\Rightarrow \text{Not every possible permutation of roots leads to a field automorphism}
        \end{align*}
    \end{proof}
    
\end{theorem}
 \hrulefill \\
\subsection{relationship between subfields, subgroups, and Automorphisms}
\begin{tikzcd}
L \arrow[d, dotted] & Aut(L:K) \arrow[d, dotted] \\
\text{intermediate fields} \arrow[d, dotted] & \text{subgroups} \\
K & {}
\end{tikzcd}

$$\Gamma := \{M : K \subseteq M \subseteq L\} \rightarrow \text{Subgroups of } Aut(L:K)$$
$$M \rightarrow \Gamma(M) = \{\sigma \in Aut(L:K) : \sigma(x) = x \text{ for all } x \in M\}$$

\hrulefill \\ 
$\Phi: \text{subgroups of } Aut(L:K) \to \text{subfields of } L/K$ \\ 
$$H \rightarrowtail \Phi(H):= \{x \in L : \sigma(x) = x \quad  \forall \sigma \in H\}$$
we see that $\Phi$ fixed subfield of $H$ \\ 
We want to show that the mapping $\Phi$ is a bijection. Note that the mapping \textit{ is not always a bijection.} When is it a bijection? \\ 

\begin{align*}
    &M \xrightarrow{\Gamma} \Gamma(M) \xrightarrow{\Phi} \Phi(\Gamma(M)) = M ?\\ 
    &H \xrightarrow{\Phi} \Phi(H) \xrightarrow{\Gamma} \Gamma(\Phi(H)) = H ?
\end{align*}
$H < Aut(L:K)$, $\Phi(H) = \{x \in L : \sigma(x) = x \quad \forall \sigma \in H\}$  \\ 
\hrulefill \\ 
$\Gamma(\Phi(H)) = \{\tau \in Aut(L:K) | \tau(x) = x \quad \forall x \in \Phi(H)\}$ 
\begin{itemize}
    \item we have a structure: subgroup $H$  $\rightarrow$ fixed subfield $\Gamma(\Phi(H))$ $\rightarrow$ subgroup that fixes the fixed subfield 
    \item $H \leq \Gamma(\Phi(H)) \leq Aut(L:K)$ 
\end{itemize}
$M \to \Gamma(M) \to \Phi(\Gamma(M)) = M$ 
\begin{itemize}
    \item subfield $\to$ subgroup  that fixes the subfield $\to$ fixed subfield of this group
    \item $M \subseteq \Phi(\Gamma(M))$
\end{itemize}

\begin{example}
    $L = \mathbb{Q}(\sqrt{2}), K = \mathbb{Q} $ and $Aut(L:K) = \mathbb{Z}_2$ $\sigma^2 = e$ \\
    \begin{tikzcd}[row sep=0.2em]
        \mathbb{Q}(\sqrt{2}) \arrow[r, "\Gamma" above] & \{e\} \arrow[l, "\Phi" below, dotted] \\[2em]
        \mathbb{Q} \arrow[r, "\Gamma" above] & \mathbb{Z}_2 \arrow[l, "\Phi" below, dotted]
    \end{tikzcd}
    right is gamma left is phi
\end{example}
\begin{example}
    $L = \mathbb{Q}(\sqrt[4]{2}), K = \mathbb{Q}$ \\
        \begin{tikzcd}[row sep=2em, column sep=4em]
        \mathbb{Q}(\sqrt[4]{2}) \arrow[r, "\Gamma" above] & \{e\} \arrow[l, "\Phi" below] \\
        \mathbb{Q}(\sqrt{2}) \arrow[rd, "\Gamma" above] & \\
        \mathbb{Q} \arrow[r, "\Gamma" below] & \mathbb{Z}_2 \arrow[lu, "\Phi" below] \arrow[l, "\Phi" above]
    \end{tikzcd}
\end{example}
\end{document}