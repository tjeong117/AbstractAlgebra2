\documentclass{article}
\usepackage{amsmath, amsthm, amssymb, amsfonts}
\usepackage{thmtools}
\usepackage{graphicx}
\usepackage{setspace}
\usepackage{cancel}
\usepackage{geometry}
\usepackage{float}
\usepackage{hyperref}
\usepackage[utf8]{inputenc}
\usepackage[english]{babel}
\usepackage{framed}
\usepackage[dvipsnames]{xcolor}
\usepackage{tcolorbox}
\usepackage{amsmath}
\usepackage{array}
\usepackage{tikz} 
\usepackage{multirow}
\usepackage{tcolorbox}
\usepackage{xcolor}
\usepackage{tikz-cd}
\usepackage{xcolor}
\usepackage{wasysym}

% Define a new color for the example box



\colorlet{LightGray}{White!90!Periwinkle}
\colorlet{LightOrange}{Orange!15}
\colorlet{LightGreen}{Green!15}

\newcommand{\HRule}[1]{\rule{\linewidth}{#1}}

\colorlet{LightGray}{black!10}
\colorlet{LightOrange}{orange!15}
\colorlet{LightGreen}{green!15}
\colorlet{LightBlue}{blue!15}
\colorlet{LightCyan}{cyan!15}



\declaretheoremstyle[name=Theorem,]{thmsty}
\declaretheorem[style=thmsty,numberwithin=section]{theorem}
\usepackage{tcolorbox} % Add missing package
\tcolorboxenvironment{theorem}{colback=LightGray}

\declaretheoremstyle[name=Definition,]{thmsty}
\declaretheorem[style=thmsty,numberwithin=section]{definition}
\tcolorboxenvironment{definition}{colback=LightBlue}

\declaretheoremstyle[name=Proposition,]{prosty}
\declaretheorem[style=prosty,numberlike=theorem]{proposition}
\tcolorboxenvironment{proposition}{colback=LightOrange}


\declaretheoremstyle[name=Example,]{prosty}
\declaretheorem[style=prosty,numberlike=theorem]{example}
\tcolorboxenvironment{example}{colback=LightOrange}

\declaretheoremstyle[name=Axiom,]{prcpsty}
\declaretheorem[style=prcpsty,numberlike=theorem]{axiom}
\tcolorboxenvironment{axiom}{colback=LightGreen}

\declaretheoremstyle[name=Lemma,]{prcpsty}
\declaretheorem[style=prcpsty,numberlike=theorem]{lemma}
\tcolorboxenvironment{lemma}{colback=LightCyan}





\setstretch{1.2}
\geometry{
    textheight=9in,
    textwidth=5.5in,
    top=1in,
    headheight=12pt,
    headsep=25pt,
    footskip=30pt
}

% ------------------------------------------------------------------------------

\begin{document}

% ------------------------------------------------------------------------------
% Cover Page and ToC
% ------------------------------------------------------------------------------

\title{ \normalsize \textsc{}
		\\ [2.0cm]
		\HRule{1.5pt} \\
		\LARGE \textbf{\uppercase{Lecture 8}}
		\HRule{2.0pt} \\ [0.6cm] \LARGE{}
		}

\date{\today}
\author{\textbf{Author} \\ 
		Tom Jeong
        }

\maketitle

\tableofcontents
\newpage

% ------------------------------------------------------------------------------
\section{L-spliting field}
L-splitting field of $x^4 - 2x^2 - 10$ over $\mathbb{Q}$
$[L:\mathbb{Q}] = ?$ \\ $\alpha_1, \alpha_2, \alpha_3, \alpha_4$ - roots of f(x) 
$$L = \mathbb{Q}(\alpha_1, \alpha_2, \alpha_3, \alpha_4) = \mathbb{Q}[\alpha_1][\alpha_2][\alpha_3][\alpha_4]$$
THe tower theorem shows that 
\begin{tikzcd}
    L = \mathbb{Q}(\alpha_1, \alpha_2, \alpha_3, \alpha_4) \arrow[d] \\
    \mathbb{Q}(\alpha_1, \alpha_2, \alpha_3) \arrow[d] \\
    \mathbb{Q}(\alpha_1, \alpha_2) \arrow[d] \\
    \mathbb{Q}(\alpha_1) \cong \mathbb{Q}[x] / \langle f(x) \rangle \arrow[d]  \\
    \mathbb{Q}
\end{tikzcd}

$[L: \mathbb{Q}] \leq n!$ where $n = \partial f$ in this case. Thus $[L:\mathbb{Q}] \mid n!$ \\ 
\begin{align*}
    4 & < [L:\mathbb{Q}] \leq 4! = 24 \\
    4 &\mid [L:\mathbb{Q}]  \mid 24 \\
    u & = x^2 \\ 
    u^2 - 2u - 10 & = 0 \\
    u & = \cfrac{2 \pm \sqrt{4 + 40}}{2} = 1 \pm \sqrt{11} \\
    \alpha_1 & = \sqrt{1 + \sqrt{11}} \\
    \alpha_2 & = -\sqrt{1 + \sqrt{11}} \\
    \alpha_3 & = \sqrt{1 - \sqrt{11}} \\
    \alpha_4 & = -\sqrt{1 - \sqrt{11}} \\
\end{align*}
$L = \mathbb{Q}(\alpha_1, \alpha_2)$
\begin{align*}
    \alpha_1^2 & = 1 + \sqrt{11} \\
    \alpha_2^2 & = 1 + \sqrt{11} \\
    -\alpha_1^2 & = -1 - \sqrt{11} \\
    2 - \alpha_1^2 & = 1 - \sqrt{11} \\
    \alpha_2^2 & = 2 - \alpha_1^2 \\
   & \text{$\alpha_2 $ is a root of g} \\ 
   x^2 - (2-\alpha_1^2) &\in \mathbb{Q}(\alpha_1) \\
   \mathbb{Q}(\alpha_1) \subseteq \mathbb{R} \\ 
   \alpha_2 &\not \in \mathbb{R} \quad \alpha_2 \not \in \mathbb{Q}(\alpha_1)
\end{align*}

\section{}
$K \cong K^\prime \quad \quad \phi: K \to K^\prime$ is isomorphism \\
$f(x) \in K[x]$ the isomorphism extends to isomorphism: $\bar{\phi}: K[x] \to K^\prime[x]$ \\ map the coefficients by $\phi$ \\ 
$f^\prime(x) \in K^\prime[x]$ \\
L-splitting field of $f(x)$ and $L^\prime$ splittting field of $f^\prime(x)$ \\ 
\begin{theorem}$L \cong L^\prime$  \leavevmode \\ 
    \begin{proof}
        induction onf $m = \partial f$ \\
        base case $m = 1$  and $\partial f= 1$ \\
        $L=K \quad L^\prime = K^\prime \quad L \cong  L^\prime$ \\
        Inductive step: 
        \\ 
        \begin{align}
            m &\rightarrow m + 1 \\ 
            \partial f & = m + 1 \\ 
            f(x) & = q_1, \dots , q_s \\ 
            &q_i-\text{irreducible} \\ 
            f^\prime = q_1^\prime, \dots , q_s^\prime \\
            K_1 &= K(x) / \langle q_1(x) \rangle \cong K_1^\prime = K^\prime(x) / \langle q_1^\prime(x) \rangle \\
        \end{align}
        \begin{tikzcd}
            L \arrow[d]& &  L^\prime \arrow[d] \\
            K_1\arrow[d] & \cong  & K_1^\prime \arrow[d] \\
            K & \cong & K^\prime
        \end{tikzcd}  \\ 
        L is the splitting field of $f(x)$ over $K_1(x)$ and $L^\prime$ is the splitting field of $f^\prime(x)$ over $K_1^\prime(x)$ \\
        L is a splitting field of $\frac{f(x)}{x - \alpha_1}$ over $K_1$ where $\alpha_1$ is a root of $q_1$ \\
        By inductive hypothesis, since $\partial(\frac{f(x)}{x - \alpha_1}) = m$, the splitting fields $L$ and $L^\prime$ are isomorphic. \\ 
        Thus, $L \cong L^\prime$ for polynomials of degree $m+1$, completing the inductive step.
     \end{proof}
    
\end{theorem}
\section{Finite fields with prime subfield} 
$\mathbb{F}$ is a finite field. Char $F = p$ p is prime. \\ 
$\mathbb{Z}_p \subseteq \mathbb{F} \quad \mathbb{Z}_p$ is prime subfield. 
\begin{enumerate}
    \item $|\mathbb{F}| = p^n$ for some $n $
    \\ $\mathbb{F} / \mathbb{Z}_p$ is a field extension
    \\ $[\mathbb{F} : \mathbb{Z}_P] = n$ (finite) then F ahs a basis over  $\mathbb{Z}_p$ \\
    \\ Then any $w \in \mathbb{F}$ has a unique expression $$ w = \sum_{i= 1}^{n} \alpha_i v_i \quad \alpha_i \in \mathbb{Z}_p$$
    p choices, n times. so number of choises = $p^n  = |\mathbb{F}|$ \\
    \item $\mathbb{F}$ is the splitting field of $x^{p^n} - x$ over $\mathbb{Z}_p$ \\ 
    $|\mathbb{F}^* | = p^n - 1 \rightarrow x \in \mathbb{F}, x \not = 0 \quad x^{p^n - 1} - 1 =0$
    \\ 
    $\rightarrow$ every element of $\mathbb{F}$ is n root of $x^{p^n} - x$ \\
    $\partial (x^{p^n} - x) = p^n$ \\ 
    $x^{p^n} - x$ has at most $p^n$  distinct roots in $\mathbb{F}$ \\
    $\rightarrow x^{p^n} - x$ splits into distinct linear factor over $\mathbb{F}$ \\
    and doesn't split over any subfield \\ 
    $\rightarrow \mathbb{F }$ is the splitting field of $x^{p^n} - x$ over $\mathbb{Z}_p$ \\
    \item The splitting field of $x^{p^n} - x$ over $\mathbb{Z}_p$ has size $p^n$ \\
\end{enumerate}
$\mathbb{F}$finite field then$\mathbb{F} = p^n$ and there is at most 1 field of size $p^n$ that has to be splitting filed of $x^{p^n} - x$ over $\mathbb{Z}_p$ \\

now we would like to show that there is exactly one field of size $p^n$ that is the splitting field of $x^{p^n} - x$ over $\mathbb{Z}_p$ \\ (property 3)

lemma for property 3.\begin{lemma}
    $f \in K(x)$ has a multiple root r iff r is a root of $f(x)$ and $f^\prime(x)$
\end{lemma}
\end{document}
