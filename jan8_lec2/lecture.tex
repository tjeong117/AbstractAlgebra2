\documentclass{article}
\usepackage{amsmath, amsthm, amssymb, amsfonts}
\usepackage{thmtools}
\usepackage{graphicx}
\usepackage{setspace}
\usepackage{cancel}
\usepackage{geometry}
\usepackage{float}
\usepackage{hyperref}
\usepackage[utf8]{inputenc}
\usepackage[english]{babel}
\usepackage{framed}
\usepackage[dvipsnames]{xcolor}
\usepackage{tcolorbox}
\usepackage{amsmath}
\usepackage{array}
\usepackage{tikz} 
\usepackage{multirow}
\usepackage{tcolorbox}
\usepackage{xcolor}
\usepackage{tikz-cd}
% Define a new color for the example box



\colorlet{LightGray}{White!90!Periwinkle}
\colorlet{LightOrange}{Orange!15}
\colorlet{LightGreen}{Green!15}

\newcommand{\HRule}[1]{\rule{\linewidth}{#1}}

\colorlet{LightGray}{black!10}
\colorlet{LightOrange}{orange!15}
\colorlet{LightGreen}{green!15}
\colorlet{LightBlue}{blue!15}
\colorlet{LightCyan}{cyan!15}



\declaretheoremstyle[name=Theorem,]{thmsty}
\declaretheorem[style=thmsty,numberwithin=section]{theorem}
\usepackage{tcolorbox} % Add missing package
\tcolorboxenvironment{theorem}{colback=LightGray}

\declaretheoremstyle[name=Definition,]{thmsty}
\declaretheorem[style=thmsty,numberwithin=section]{definition}
\tcolorboxenvironment{definition}{colback=LightBlue}

\declaretheoremstyle[name=Proposition,]{prosty}
\declaretheorem[style=prosty,numberlike=theorem]{proposition}
\tcolorboxenvironment{proposition}{colback=LightOrange}


\declaretheoremstyle[name=Example,]{prosty}
\declaretheorem[style=prosty,numberlike=theorem]{example}
\tcolorboxenvironment{example}{colback=LightOrange}

\declaretheoremstyle[name=Axiom,]{prcpsty}
\declaretheorem[style=prcpsty,numberlike=theorem]{axiom}
\tcolorboxenvironment{axiom}{colback=LightGreen}

\declaretheoremstyle[name=Lemma,]{prcpsty}
\declaretheorem[style=prcpsty,numberlike=theorem]{lemma}
\tcolorboxenvironment{lemma}{colback=LightCyan}





\setstretch{1.2}
\geometry{
    textheight=9in,
    textwidth=5.5in,
    top=1in,
    headheight=12pt,
    headsep=25pt,
    footskip=30pt
}

% ------------------------------------------------------------------------------

\begin{document}

% ------------------------------------------------------------------------------
% Cover Page and ToC
% ------------------------------------------------------------------------------

\title{ \normalsize \textsc{}
		\\ [2.0cm]
		\HRule{1.5pt} \\
		\LARGE \textbf{\uppercase{Lecture 1}}
		\HRule{2.0pt} \\ [0.6cm] \LARGE{}
		}

\date{\today}
\author{\textbf{Author} \\ 
		Tom Jeong
        }

\maketitle

\tableofcontents
\newpage

% ------------------------------------------------------------------------------
\section{D-integral domains}
Q(D)-fields of faction, smalliest field containing Define

\underline{universa, property}
theta: D -> Q theta injective, k- field
(draw down i: D to Q(D))  sending a to (a,1) 

and draw phi D(D) to K 
and say theta = phi composite i 
\\ 

\begin{tikzcd}[column sep=huge, row sep=huge]
    D \arrow[r, "\theta", dashed] \arrow[d, "i"'] & K \\
    Q(D) \arrow[ur, "\phi"'] &
    \end{tikzcd}
    \begin{itemize}
        \item $i: D \to Q(D)$ sends $a \mapsto (a,1)$
        \item $\theta$ is injective
        \item $K$ is a field
        \item $\theta = \phi \circ i$
        \end{itemize}    


 
        \section{Characteristic of a Ring}

        Let $R$ be a unital commutative ring. The characteristic of $R$, denoted $\operatorname{char}(R)$, is defined as follows:
        
        \begin{definition}
        The characteristic of a ring $R$ is the smallest positive integer $n$ such that
        \[ n \cdot 1_R = \underbrace{1_R + 1_R + \cdots + 1_R}_{n \text{ times}} = 0_R \]
        If no such positive integer exists, we say $\operatorname{char}(R) = 0$.
        \end{definition}
        
        \begin{proposition}
        For a unital commutative ring $R$, exactly one of the following holds:
        \begin{enumerate}
            \item $\operatorname{char}(R) = 0$: In this case, the additive subgroup generated by $1_R$ is infinite.
            \item $\operatorname{char}(R) = n > 0$: In this case, $n$ is the smallest positive integer such that $n \cdot 1_R = 0_R$.
        \end{enumerate}
        \end{proposition}
        
       
        If $k \cdot 1_R = m \cdot 1_R$ for some integers $k,m$, then:
        \[ (k-m) \cdot 1_R = 0_R \]
        This means that if $\operatorname{char}(R) = n > 0$, then $n$ divides $k-m$.
  
        
        example:
        \begin{enumerate}
            \item $\operatorname{char}(\mathbb{Z}) = 0$
            \item $\operatorname{char}(\mathbb{Q}) = 0$
            \item $\operatorname{char}(\mathbb{F}_p) = p$ for any prime field
            \item For any field $K$, $\operatorname{char}(K)$ is either $0$ or a prime number
        \end{enumerate}
      
        
        \begin{proposition}
        If $R$ is a domain (i.e., has no zero divisors), then $\operatorname{char}(R)$ is either $0$ or prime.
        \end{proposition}
\begin{proof}
    \leavevmode \\ 
    s = char(k) and $s = ab$ where $a,b <s$  \\ 
    $(a\cdot 1)(b\cdot 1) = (a\cdot b) \cdot 1 = 0 \rightarrow a\cdot 1 = 0 \text{ or } b\cdot 1 = 0 \text{ but } a,b < s$
\end{proof}

k-field:: $char(K) = 0 \rightarrow \mathbb{Q } \subseteq K $ \\ 
$char(K) = p \rightarrow \mathbb{Z}_p \text{ or } \mathbb{F_p} \subseteq K$ \\ 
$\mathbb{Q}$ or $\mathbb{Z}_p$ are called prime subfields of K.  \\ 
\noindent\rule{\textwidth}{0.5pt}

\section{vector Space} 
\begin{proposition}
    Lets say a field is inside another field, $F \subseteq K$ then $K$ is an $F$-vector space 

\end{proposition}

So vector space over $F$ ( F field) if  \begin{enumerate}
    \item $s_1 s_2 \in S \rightarrow s_1 + s_2 \in S$
    \item $c\cdot s_1 \in S, c\in F$ 
    \item $c(s_1 + s_2) = cs_1 + cs_2$ 
\end{enumerate} 

$\mathbb{R} \subseteq \mathbb{C}$ 
\begin{align*} 
    & a + bi \\ 
    &\text{1, i is a basis of }\mathbb{C } \text{ over } \mathbb{R} \\ 
    &\mathbb{C} \cong \mathbb{R}^2 \text{ but then } \\ 
    &\mathbb{Q} \subseteq \mathbb{R} \text{ .. } \mathbb{R} \text{ is an infinite-dimensional vector space over } \mathbb{Q}
\end{align*}

K is a field extension of F 
\begin{proposition}[freshmans dream ] \leavevmode \\ 
    if $char(K) = p, K $ field then, $(x+ y)^p = x^p + y^p$
\end{proposition}
\begin{proof}
    binomial expansion of 
    \begin{align*}
        (x + y)^p &= x^p + \binom{p}{1} x^{p-1}y + \binom{p}{2} x^{p-2}y^2 + \cdots + \binom{p}{p-1} xy^{p-1} + y^p \\
        &= x^p + \cancelto{0}{\binom{p}{1}} x^{p-1}y + \cancelto{0}{\binom{p}{2}} x^{p-2}y^2 + \cdots + \cancelto{0}{\binom{p}{p-1}} xy^{p-1} + y^p \\
        &= x^p + y^p
    \end{align*}

        In characteristic $p$, all binomial coefficients $\binom{p}{k}$ for $1 \leq k \leq p-1$ are divisible by $p$, hence equal to zero in the field.
\end{proof}
Let $K$ be a field of characteristic $p > 0$. The Frobenius homomorphism $\phi: K \to K$ is defined as:

\begin{align*}
    \phi: K &\to K \\
    x &\mapsto x^p
\end{align*}

\begin{proposition}[Properties of Frobenius]
The map $\phi$ is a ring homomorphism:
\begin{enumerate}
    \item $\phi(x + y) = (x + y)^p = x^p + y^p = \phi(x) + \phi(y)$ 
        \quad (using the binomial expansion in char $p$)
    \item $\phi(xy) = (xy)^p = x^p y^p = \phi(x)\phi(y)$
    \item $\phi(1) = 1^p = 1$
\end{enumerate}
\end{proposition}

\begin{example}
In $\mathbb{F}_p$, the Frobenius map is the identity map since:
\[ a^p = a \text{ for all } a \in \mathbb{F}_p \]
This is known as Fermat's Little Theorem.
\end{example}
 
Goal Theorem: 
\begin{theorem}\leavevmode \\ 
    K-field, K(x) - polynomial rings. 
    
    \begin{enumerate}
        \item For any polynomials $f,g \in K[x]$, there exists a greatest common divisor $d \in K[x]$ such that:
        \[ d = af + bg \quad \text{for some } a,b \in K[x] \]
        This is known as Bézout's identity in $K[x]$.
        
        \item $K[x]$ is a Principal Ideal Domain (PID).
        
        \item $K[x]$ is a Unique Factorization Domain (UFD).
        
        \item For any polynomial $f(x) \in K[x]$, the following are equivalent:
        \begin{enumerate}
            \item $f(x)$ is irreducible in $K[x]$
            \item The quotient ring $K[x]/\langle f(x) \rangle$ is a field
    \end{enumerate}
\end{enumerate}
\end{theorem}
\begin{proof}
    $ K \subseteq K[x] / \langle f(x) \rangle  $
\end{proof}
\section{Euclidian Domain}
A integral domain  $D$ is a Euclidian Domain (ED) if there exists a function \begin{align*}
    &\delta: R \to \mathbb{Z}_{\geq 0}  \text { st } \\ 
    & \delta(0) = 0
\end{align*}
and for all $a \in D, b \in D^* = D \backslash \{0\}$, \\  there exists $g, r \in  D$ such that $a = qb = r$  \\ AND $\delta(r) \leq \delta(b)$ \\ 

 This allows us t define division with remainder. \\
$\delta^{-1}(0) = 0$
 \\ $\delta(b) = 0, b \not = 0$ \\ 
 $a = qb + r \rightarrow \delta (r) < \delta(b) \rightarrow \leftarrow$  \\ 
 example: $\mathbb{Z}, \delta(r) =|r| $ \\

 
 \begin{definition}[PID]
    D-integral domain is a PID, if all ideals in D are principal, 

    generated by one element 
 \end{definition}

 \begin{proposition}
    every euclidian domain is a PID 
 \end{proposition}
 \begin{proof} \leavevmode \\ 
    $\{0\}$ principal $\langle 0 \rangle$  \\ 
    $D = \langle 1 \rangle$ \\ 
    I - proper ideal of D: wts a single element that generates all of I. let $b \in I$ be the element with the smallest positive $\delta$ \\ 
    then we would like to claim that $I = \langle b \rangle$ \\ 

    suppose that $a \in I$. Then $a = qb +r$ wgere $\delta(r) < \delta(b)$ \\ 
    $r = a -   qb$ a, qb in Ideal, thus r is in ideal. 
    \\ $\delta(r)$ must be 0 ($ r = 0$) since b is the smallest element.   \\ 
    $\therefore a\in \langle b \rangle$ 
 \end{proof}

 In PID there is a well-defined $gcd(a,b)$ where $a,b \in D$. 

 \underline{And}: $d = gcd(a,b) \rightarrow d = af + bg \text{ where } f,g \in D$  \\ 

 \begin{proof}
    \begin{align*}
        &a, b \in D \\ 
        &\langle a, b \rangle = \langle d \rangle  \\ 
        & d = gcd(a, b) \\ 
        & gcd(a,b) \text{ is only depende oup to units.  since d in ideal of a b}
    \end{align*}
 \end{proof}
 example: $\mathbb{Z} $ and 8, 12

 \end{document}
